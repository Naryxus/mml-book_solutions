\documentclass[solution]{tudexercise}
\usepackage[utf8]{inputenc}
\usepackage[english]{babel}
\usepackage{amsmath}

%\setlength{\parindent}{0pt}

\title{Mathematics for Machine Learning}
\subtitle{Solution of the Exercises}
\subsubtitle{Stefan Thaut (1800351)\\
			Department 20 - Computer Science\\
			\today}
\author{Stefan Thaut}
\date{\today}

\newcommand{\qed}{\hspace*{\fill}$\square$}

\begin{document}
	\maketitle
	
	\setcounter{section}{1}
	\section{Linear Algebra}
		\subsection{}
			\subsubsection{}
				\begin{enumerate}
				\item
				We have to show, that $\mathbb{R} \setminus \{-1\}$ is closed under $*$, the associativity, the existence of a neutral and inverse elements and the commutativity.\\
				For the closure of $\mathbb{R} \setminus \{-1\}$ we can use the closure of the addition and multiplication in $\mathbb{R}$. Then we have to show that there are no $a$ and $b$ in $\mathbb{R} \setminus \{-1\}$, so that $a * b = -1$.\\\\
				Assuming that $\exists a, b \in \mathbb{R} \setminus \{-1\}$ with $a * b = -1$. Then it is
				\begin{alignat*}{2}
&&a * b = ab + a + b &= -1\\
\Longleftrightarrow\quad &&ab + a &= -1 -b\\
\Longleftrightarrow\quad &&a(b + 1) &= -(b + 1)\\
\Longleftrightarrow\quad &&a &= -\frac{b + 1}{b + 1} = -1
				\end{alignat*}
				So we got a contradiction and that shows that there are no $a, b \in \mathbb{R} \setminus \{-1\}$ so that $a * b = -1$.\\
				Consider $a, b, c \in \mathbb{R} \setminus \{-1\}$. Then it is
				\begin{align*}
(a * b) * c &= (ab + a + b) * c\\
&= (ab + a + b)c + (ab + a + b) + c\\
&= abc + ac + bc + ab + a + b + c\\
&= abc + ab + ac + a + bc + b + c\\
&= a(bc + b + c) + a + (bc + b + c)\\
&= a * (bc + b + c) = a * (b * c)\\
				\end{align*}
				That shows the associativity of $*$.
				\newpage
				The neutral element is $0$, because: 
				\begin{align*}
a * 0 &= a \cdot 0 + a + 0 = 0 + a + 0 = a\ \text{and}\\
0 * a &= 0 \cdot a + 0 + a = 0 + 0 + a = a
				\end{align*}
				for any $a \in \mathbb{R} \setminus \{-1\}$.\\
				Consider $a^{-1} = -a / (a + 1)$. Then it is
				\begin{align*}
a * a^{-1} &= a * - \frac{a}{a + 1}\\
&= a(- \frac{a}{a + 1}) + a + (- \frac{a}{a + 1})\\
&= \frac{-a^2}{a + 1} + a - \frac{a}{a + 1}\\
&= \frac{-a^2 - a}{a + 1} + \frac{a(a + 1)}{a + 1}\\
&= \frac{-a^2 - a}{a + 1} + \frac{a^2 + a}{a + 1} = 0
				\end{align*}
				The proof of $a^{-1} * a = 0$ works analogously.\\
				The proof of the commutativity is straight forward and based on the commutativity of the addition and multiplication in $\mathbb{R}$. Consider $a, b \in \mathbb{R} \setminus \{-1\}$. Then it is
				\begin{align*}
a * b = ab + a + b = ba + b + a = b * a
				\end{align*}
				So we have shown all axioms of an Abelian group. \qed
				
				\item
				It is
				\begin{align*}
3 * x * x &= (3x + 3 + x) * x\\
&= (4x + 3) * x\\
&= (4x + 3)x + (4x + 3) + x\\
&= 4x^2 + 3x + 4x + 3 + x = 4x^2 + 8x + 3
				\end{align*}
				We can now solve the quadtratic formula $4x^2 + 8x + 3 = 15 \Longleftrightarrow 4x^2 + 8x - 12 = 0$ using the completing the square method proposed by \textsc{Hoehn} in \cite{hoehn1975more}:
				\begin{align*}
x &= \frac{-8 \pm \sqrt{8^2 - 4 \cdot 4 \cdot (-12)}}{2 \cdot 4}\\
&= \frac{-8 \pm \sqrt{64 + 192}}{8}\\
&= \frac{-8 \pm \sqrt{256}}{8}\\
&= \frac{-8 \pm 16}{8} = -1 \pm 2
				\end{align*}
				Thus the solution of the equation is $x_1 = -3$ and $x_2 = 1$.
				\end{enumerate}
	
	\bibliography{build/bibliography}
	\bibliographystyle{plain}

\end{document}