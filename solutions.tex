\documentclass[solution]{tudexercise}
\usepackage[utf8]{inputenc}
\usepackage[english]{babel}
\usepackage{amsmath}

%\setlength{\parindent}{0pt}

\title{Mathematics for Machine Learning}
\subtitle{Solution of the Exercises}
\subsubtitle{Stefan Thaut (1800351)\\
			Department 20 - Computer Science\\
			\today}
\author{Stefan Thaut}
\date{\today}

\newcommand{\qed}{\hspace*{\fill}$\square$}
\renewcommand{\labelenumi}{\arabic{enumi}.}

\begin{document}
	\maketitle
	
	\setcounter{section}{1}
	\section{Linear Algebra}
		\subsection{}
			\subsubsection{}
				\begin{enumerate}
				\item
				We have to show, that $\mathbb{R} \setminus \{-1\}$ is closed under $*$, the associativity, the existence of a neutral and inverse elements and the commutativity.\\
				For the closure of $\mathbb{R} \setminus \{-1\}$ we can use the closure of the addition and multiplication in $\mathbb{R}$. Then we have to show that there are no $a$ and $b$ in $\mathbb{R} \setminus \{-1\}$, so that $a * b = -1$.\\\\
				Assuming that $\exists a, b \in \mathbb{R} \setminus \{-1\}$ with $a * b = -1$. Then it is
				\begin{alignat*}{2}
&&a * b = ab + a + b &= -1\\
\Longleftrightarrow\quad &&ab + a &= -1 -b\\
\Longleftrightarrow\quad &&a(b + 1) &= -(b + 1)\\
\Longleftrightarrow\quad &&a &= -\frac{b + 1}{b + 1} = -1
				\end{alignat*}
				So we got a contradiction and that shows that there are no $a, b \in \mathbb{R} \setminus \{-1\}$ so that $a * b = -1$.\\
				Consider $a, b, c \in \mathbb{R} \setminus \{-1\}$. Then it is
				\begin{align*}
(a * b) * c &= (ab + a + b) * c\\
&= (ab + a + b)c + (ab + a + b) + c\\
&= abc + ac + bc + ab + a + b + c\\
&= abc + ab + ac + a + bc + b + c\\
&= a(bc + b + c) + a + (bc + b + c)\\
&= a * (bc + b + c) = a * (b * c)\\
				\end{align*}
				That shows the associativity of $*$.
				\newpage
				The neutral element is $0$, because: 
				\begin{align*}
a * 0 &= a \cdot 0 + a + 0 = 0 + a + 0 = a\ \text{and}\\
0 * a &= 0 \cdot a + 0 + a = 0 + 0 + a = a
				\end{align*}
				for any $a \in \mathbb{R} \setminus \{-1\}$.\\
				Consider $a^{-1} = -a / (a + 1)$. Then it is
				\begin{align*}
a * a^{-1} &= a * - \frac{a}{a + 1}\\
&= a(- \frac{a}{a + 1}) + a + (- \frac{a}{a + 1})\\
&= \frac{-a^2}{a + 1} + a - \frac{a}{a + 1}\\
&= \frac{-a^2 - a}{a + 1} + \frac{a(a + 1)}{a + 1}\\
&= \frac{-a^2 - a}{a + 1} + \frac{a^2 + a}{a + 1} = 0
				\end{align*}
				The proof of $a^{-1} * a = 0$ works analogously.\\
				The proof of the commutativity is straight forward and based on the commutativity of the addition and multiplication in $\mathbb{R}$. Consider $a, b \in \mathbb{R} \setminus \{-1\}$. Then it is
				\begin{align*}
a * b = ab + a + b = ba + b + a = b * a
				\end{align*}
				So we have shown all axioms of an Abelian group. \qed
				
				\item
				It is
				\begin{align*}
3 * x * x &= (3x + 3 + x) * x\\
&= (4x + 3) * x\\
&= (4x + 3)x + (4x + 3) + x\\
&= 4x^2 + 3x + 4x + 3 + x = 4x^2 + 8x + 3
				\end{align*}
				We can now solve the quadtratic formula $4x^2 + 8x + 3 = 15 \Longleftrightarrow 4x^2 + 8x - 12 = 0$ using the completing the square method proposed by \textsc{Hoehn} in \cite{hoehn1975more}:
				\begin{align*}
x &= \frac{-8 \pm \sqrt{8^2 - 4 \cdot 4 \cdot (-12)}}{2 \cdot 4}\\
&= \frac{-8 \pm \sqrt{64 + 192}}{8}\\
&= \frac{-8 \pm \sqrt{256}}{8}\\
&= \frac{-8 \pm 16}{8} = -1 \pm 2
				\end{align*}
				Thus the solution of the equation is $x_1 = -3$ and $x_2 = 1$.
				\end{enumerate}
			
			\subsubsection{}
				\begin{enumerate}
				\item
				At first we have to be careful, because $(\mathbb{Z}_n, \oplus)$ would not be a group with the unmodified given mapping, because $\mathbb{Z}_n$ would not be closed under $\oplus$: Let $n = 3$, then $\mathbb{Z}_3 = \{\overline{0}, \overline{1}, \overline{2}\}$. So consider $\overline{1}, \overline{2} \in \mathbb{Z}_3$, then:
				\begin{align*}
\overline{1} \oplus \overline{2} = \overline{1 + 2} = \overline{3} \notin \mathbb{Z}_3
				\end{align*}
				Thus we have to modify the mapping by adding a modulo to the addition:
				\begin{align}
\overline{a} \oplus \overline{b} = \overline{a + b \mod n}
				\end{align}
				Now $\mathbb{Z}_n$ is closed under $\oplus$.\\
				\begin{itemize}
				\item[(Associativity)] Let $\overline{a}, \overline{b}, \overline{c} \in \mathbb{Z}_n$. Then it is
				\begin{align*}
(\overline{a} \oplus \overline{b}) \oplus \overline{c} &= \overline{a + b \mod n} \oplus \overline{c}\\
&= \overline{(a + b \mod n) + c \mod n}\\
&= \overline{a + b + c \mod n}\\
&= \overline{a \mod n + (b + c \mod n)}\\
&= \overline{a} \oplus \overline{b + c \mod n} = \overline{a} \oplus (\overline{b} \oplus \overline{c})\\
				\end{align*}
				
				\item[(Neutral Element)]
				The neutral element is $\overline{0} \in \mathbb{Z}_n$. Let $\overline{a} \in \mathbb{Z}_n$, then it is:
				\begin{align*}
\overline{a} \oplus \overline{0} = \overline{a + 0 \mod n} = \overline{a \mod n} = \overline{a}
				\end{align*}
				and
				\begin{align*}
\overline{0} \oplus \overline{a} = \overline{0 + a \mod n} = \overline{a \mod n} = \overline{a}
				\end{align*}
				
				\item[(Inverse Element)]
				Let $\overline{a} \in \mathbb{Z}_n$. Then the inverse element of $\overline{a}$ is $\overline{a}^{\,-1} = \overline{n - a}.$
				It is
				\begin{align*}
\overline{a} \oplus \overline{a}^{\,-1} &= \overline{a} \oplus \overline{n - a}\\
&= \overline{a + (n - a) \mod n}\\
&= \overline{n \mod n} = \overline{0}\\
				\end{align*}
				The proof of $\overline{a}^{\,-1} \oplus \overline{a}$ works analogously.
				
				\item[(Commutativity)]
				For the proof of the commutativity in $\mathbb{Z}_n$ we use the commutativity of the addition in $\mathbb{Z}$. Let $\overline{a}, \overline{b} \in \mathbb{Z}_n$. Then it is
				\begin{align*}
\overline{a} \oplus \overline{b} = \overline{a + b \mod n} = \overline{b + a \mod n} = \overline{b} \oplus \overline{a}
				\end{align*}
				\end{itemize}
				We have shown, that $(\mathbb{Z}_n, \oplus)$ is an Abelian Group.\qed
				
				\item
				Firstly we also have to modify the mapping for mathematical correctness:
				\begin{align}
\overline{a} \otimes \overline{b} = \overline{a \cdot b \mod n}
				\end{align}
				\begin{table}[h]
					\centering
					\caption{The times table of $\mathbb{Z}_5 \setminus \{\overline{0}\}$ under $\otimes$}
					\label{tab:timesTableZ5}
					\begin{tabular}{c|cccc}
$\otimes$ & $\overline{1}$ & $\overline{2}$ & $\overline{3}$ & $\overline{4}$\\\hline
$\overline{1}$ & $\overline{1}$ & $\overline{2}$ & $\overline{3}$ & $\overline{4}$\\
$\overline{2}$ & $\overline{2}$ & $\overline{4}$ & $\overline{1}$ & $\overline{3}$\\
$\overline{3}$ & $\overline{3}$ & $\overline{1}$ & $\overline{4}$ & $\overline{2}$\\
$\overline{4}$ & $\overline{4}$ & $\overline{3}$ & $\overline{2}$ & $\overline{1}$\\
					\end{tabular}
				\end{table}
				In table \ref{tab:timesTableZ5} we can see that for each elements $\overline{a}, \overline{b} \in \mathbb{Z}_5 \setminus \{\overline{0}\}$ the result of $\overline{a} \otimes \overline{b} \in \mathbb{Z}_5 \setminus \{\overline{0}\}$. The neutral element of this mapping is $\overline{1}$, because $\overline{a} \otimes \overline{1} = \overline{a \cdot 1 \mod 5} = \overline{a}$. We can read the inverse elements of each element in $\mathbb{Z}_5 \setminus \{\overline{0}\}$ out of table \ref{tab:timesTableZ5}, where the result is $\overline{1}$. And the commutativity follows directly from the commutativity of the multiplication in $\mathbb{Z}$.\qed
				
				\item
				Consider $\overline{4} \in \mathbb{Z}_8 \setminus \{\overline{0}\}$. If we multiply $\overline{4}$ with all elements of $\mathbb{Z}_8 \setminus \{\overline{0}\}$, we get the following results:
				\begin{align*}
\overline{4} \otimes \overline{1} &= \overline{4} & \overline{4} \otimes \overline{2} &= \overline{0} & \overline{4} \otimes \overline{3} &= \overline{4} & \overline{4} \otimes \overline{4} &= \overline{0}\\
\overline{4} \otimes \overline{5} &= \overline{4} & \overline{4} \otimes \overline{6} &= \overline{0} & \overline{4} \otimes \overline{7} &= \overline{4}\\
				\end{align*}
				Hence we see, that there is no inverse element of $\overline{4}$ in $\mathbb{Z}_8 \setminus \{\overline{0}\}$. Thus $(\mathbb{Z}_8 \setminus \{\overline{0}\})$ is not a group.\qed
				
				\item
				The key point for showing that $(\mathbb{Z}_n \setminus \{\overline{0}\}, \otimes)$ is a group, is to show the existence of an inverse element for each element $\overline{a} \in \mathbb{Z}_n \setminus \{\overline{0}\}$.\\
				Assuming $n \in \mathbb{N} \setminus \{0\}$ is prime. Hence $a$ and $n$ are relatively prime and with that we know from Bézout theorem that there are two integers $u$ and $v$ such that $au + nv = 1$. That implies $\overline{au} \otimes \overline{nv} = \overline{1}$ and also $\overline{a} \otimes \overline{u} = \overline{1}$. Thus the inverse element of $\overline{a}$ is $\overline{u}$.\\
				Assuming $(\mathbb{Z}_n \setminus \{\overline{0}\}, \otimes)$ is a group. Also assuming $n$ is not prime, i.e. $\exists k, l \in \mathbb{N}$ such that $1 < k, l < n$ and $n = kl$. We see that $\overline{k}, \overline{l} \in \mathbb{Z}_n \setminus \{\overline{0}\}$. But it is $\overline{k} \otimes \overline{l} = \overline{k \cdot l \mod n} = \overline{n \mod n} = \overline{0} \notin \mathbb{Z}_n \setminus \{\overline{0}\}$. Hence $\mathbb{Z}_n \setminus \{\overline{0}\}$ would not be closed under $\otimes$ and would therefore be no group. Thus $n$ has to be prime.\qed
				\end{enumerate}
	
	\bibliography{build/bibliography}
	\bibliographystyle{plain}

\end{document}